\noindent
\textbf{Bolzano-Weierstrass Theorem} A set \(A \subset R^d\) 
is compact if and only if it is closed and bounded in \(R^d\).
\\
\rule{\linewidth}{0.1mm}

\noindent
\textbf{Tychonoff theorem} If \(A\) and \(B\) are compact sets, 
so is their Cartesian product.
\\
\rule{\linewidth}{0.1mm}

\noindent
\textbf{Lemma} A set is convex if and only if it contains all convex 
combinations of its points.
\\
\rule{\linewidth}{0.1mm}

\noindent
\textbf{Convex hull} Set of all convex combinations of points of a set.
\\
\rule{\linewidth}{0.1mm}

\noindent
\textbf{Gauss-Lucas Theorem} Let \(f(z)\) be an \(n^{th}\) degree real 
polynomial. Let \(z_1, z_2, \ldots, z_n\) be the roots of \(f(z)\). 
Then \(n-1\) roots of \(f'(z)\) lies in 
\(conv(\{z_1,z_2,\ldots,z_n\})\).
\\
\rule{\linewidth}{0.1mm}

\noindent
\textbf{Caratheodory Theorem} Let \(S \subset R^d\). Then any 
\(x \in conv(S)\) can be represented as a convex combination of \(d+1\) 
points of \(S\).
\\
\rule{\linewidth}{0.1mm}

\noindent
\textbf{Lemma} If \(A \subset R^d\) is convex, then \(A^o = int(A)\) is 
convex.
\\
\rule{\linewidth}{0.1mm}

\noindent
\textbf{Lemma} If \(A \subset R^d\) is compact, then \(conv(A)\) is 
compact.
\\
\rule{\linewidth}{0.1mm}

\noindent
\textbf{Lemma} Let \(A \subset R^n\) and \(u_0 \in int(A)\) be an 
interior point. Then for any \(u_1 \in A\),
\({u_\alpha = (1-\alpha)u_0 + \alpha u_1 \in int(A) 
\quad 0 \le \alpha < 1. }\)
\\
\rule{\linewidth}{0.1mm}

% \noindent
% \textbf{Lemma} If \(A \subset R^n\) is convex, then \(A^o = int(A)\) is 
% convex.
% \\
% \rule{\linewidth}{0.1mm}

\noindent
\textbf{Lemma} If \(\{x_1,x_2,\ldots,x_{d+1}\} \subset R^d\) are affinely 
independent points, then the convex hull 
\(conv(\{x_1,x_2,\ldots,x_{d+1}\})\) has a 
non-zero interior.
\\
\rule{\linewidth}{0.1mm}

\noindent
\textbf{Theorem} Let \(A \subset R^d\) be a convex set. If 
\(int(A) = \phi\), then there exists an affine subspace 
\(L \subset R^d\) such that \(A \subset L\) and \(dim(L) < d\).
\\
\rule{\linewidth}{0.1mm}

\noindent
\textbf{Definition} \(F\) is a face of a closed convex set if there 
exists a hyperplane \(H\) such that \(H\) isolates \(C\) and 
\(F = H \cap C\). If \(F\) is a point, then \(F\) is called an 
exposed point. A non-empty face \(F \ne C\) is called a proper face.
\\
\rule{\linewidth}{0.1mm}

\noindent
\textbf{Lemma} Let \(K \subset R^d\) be a convex set with non-empty 
interior. Let \(U \in \partial K\) be a point on the boundary. 
Then there exists an 
affine hyperplane \(H\) passing through \(U\) such that \(H\) isolates 
\(K\). \(H\) is called a supporting hyperplane.
\\
\rule{\linewidth}{0.1mm}

\noindent
\textbf{Lemma} Let \(K \subset R^d\) be a closed convex set. Let 
\(U \in \partial K\). Then there exists a proper face \(F\) of \(K\) 
such that \(U \in F\).
\\
\rule{\linewidth}{0.1mm}

\noindent
\textbf{Definition} Let \(A \subset R^d\). \(x \in A\) is an extreme 
point of \(A\), if \(x = (a+b)/2, a \in A,b \in A\) implies 
\(x = a = b\).
\\
\rule{\linewidth}{0.1mm}

\noindent
\textbf{Lemma} A compact set has at least one extreme point.
\\
\rule{\linewidth}{0.1mm}

\noindent
\textbf{Krein-Millman Theorem} Let \(K \subset R^d\) be a compact 
convex set. Then \(K = conv(ex(K))\), where \(ex(K)\) is the set of 
extreme points of \(K\).
\\
\rule{\linewidth}{0.1mm}

\noindent
\textbf{Lemma} Let \(K \subset R^d\) be a compact convex set. Then the 
linear function \(f(x) = a^T x :R^d \rightarrow R\) attains a maximum 
or minimum on an extreme point of \(K\).
\\
\rule{\linewidth}{0.1mm}

\noindent
\textbf{Birkhoff-von Neumann Theorem} The vertices of the Birkhoff 
polytope \(B_n\) are exactly the \(n \times n\) permutation matrices. 
The polyhedron \(B_n\) of all \(n \times n\) doubly stochastic matrices 
is called the Birkhoff polytope.
\\
\rule{\linewidth}{0.1mm}

\noindent
\textit{Convex set}
\begin{align*}
  \theta x + (1-\theta) y \in C, \quad 0 \le \theta \le 1
\end{align*}
\rule{\linewidth}{0.1mm}

\noindent
\textit{Affine set}
\begin{align*}
  \theta x + (1-\theta) y \in L, \quad \theta \in R
\end{align*}
\rule{\linewidth}{0.1mm}

\noindent
\textit{Cone}
\begin{align*}
  \theta_1 x + \theta_2 y \in K, \quad \theta_1, \theta_2 \ge 0
\end{align*}
\textit{Proper Cone}: closed, convex, solid, pointed\\
\rule{\linewidth}{0.1mm}

\noindent
\textit{Dual cone}
\begin{align*}
  K^* = \{y \, | \, x^T y \ge 0 \mbox{ for all } x \in K\}
\end{align*}
\rule{\linewidth}{0.1mm}
% \noindent

% \rule{\linewidth}{0.1mm}

% \noindent

% \rule{\linewidth}{0.1mm}

% \noindent

% \rule{\linewidth}{0.1mm}